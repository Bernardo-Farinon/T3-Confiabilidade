\documentclass[12pt]{article}
\usepackage[a4paper,margin=2cm]{geometry}
\usepackage{graphicx}
\usepackage{titlesec}
\usepackage{setspace}
\usepackage{indentfirst}
\usepackage{listings}
\usepackage{graphicx}

\onehalfspacing
\setlength{\parindent}{1.25cm}

\title{\textbf{Relatório — JTAG e SWD com J-Link v9 no STM32F072RB}}
\author{Bernardo Farinon}
\date{}

\begin{document}
\maketitle

\section{Introdução}

Este relatório apresenta um estudo teórico e prático sobre o funcionamento do padrão \textbf{JTAG} (IEEE~1149.1) e sua alternativa \textbf{SWD} (Serial Wire Debug), amplamente utilizada em microcontroladores ARM.

O trabalho inclui:

\begin{itemize}
    \item uma explicação detalhada da arquitetura interna do JTAG, seus registradores e sinais de operação;
    \item uma análise conceitual do SWD e suas diferenças em relação ao JTAG;
    \item uma demonstração prática utilizando um \textbf{J-Link v9} conectado à placa \textbf{STM32F072RB}, explorando leitura de registradores, controle da execução e manipulação da memória.
\end{itemize}

\section{Motivação para JTAG}

A complexidade crescente dos circuitos integrados tornou extremamente difícil a inspeção física com sondas tradicionais, devido ao grande número de pinos e às interconexões internas não acessíveis externamente. Isso motivou a criação de um padrão capaz de:

\begin{itemize}
    \item testar interconexões internas sem acesso físico;
    \item verificar soldas e continuidade em placas complexas;
    \item depurar sistemas embarcados sem interferir no funcionamento da CPU;
    \item padronizar o acesso a dispositivos durante testes de fábrica.
\end{itemize}

Assim surgiu o \textbf{JTAG — Joint Test Action Group}, oferecendo um mecanismo de acesso eletrônico interno ao chip para fins de teste e depuração.

\section{Arquitetura e Funcionamento do JTAG}

A interface JTAG utiliza 4--5 sinais externos (TCK, TMS, TDI, TDO e TRST) e controla uma estrutura interna composta por registradores de instrução, registradores de dados e uma máquina de estados denominada \textbf{TAP Controller}. A Figura~\ref{fig:jtag_arch} ilustra essa arquitetura integrada ao chip.

\begin{figure}[h!]
    \centering
    \includegraphics[width=0.95\textwidth]{jtagchip.png}
    \caption{Arquitetura interna do JTAG no chip}
    \label{fig:jtag_arch}
\end{figure}

\subsection{Descrição da Figura}

Na imagem apresentada:

\begin{itemize}
    \item As regiões em \textbf{verde} representam o \textbf{Boundary Scan}, composto por várias Boundary Scan Cells conectadas em série ao redor dos pinos físicos do chip. Esse conjunto forma o \textit{Boundary Scan Register}, utilizado para teste fisico de placas.
    \item As regiões em \textbf{vermelho} (``Debug CTRL'' e ``Flash CTRL'') correspondem a módulos internos da CPU que podem ser acessados via JTAG, dependendo da implementação. Esses blocos permitem depuração e programação da memória Flash.
    \item A região em \textbf{rosa} representa o \textbf{TAP (Test Access Port)}, contendo:
    \begin{itemize}
        \item o \textbf{TAP Controller} (máquina de estados controlada por TMS);
        \item o \textbf{Instruction Register (IR)};
        \item os \textbf{Data Registers (DR)}, como BYPASS, IDCODE e o próprio Boundary Scan.
    \end{itemize}
\end{itemize}

O TAP atua como ``roteador'' interno, conectando o caminho \textbf{TDI → IR/DR → TDO} conforme a instrução selecionada.

\subsection{Sinais JTAG}

\begin{itemize}
    \item \textbf{TCK}: clock da interface.
    \item \textbf{TMS}: controla as transições da máquina de estados.
    \item \textbf{TDI}: entrada serial de instruções e dados.
    \item \textbf{TDO}: saída serial de instruções e dados.
    \item \textbf{TRST} (opcional): reset assíncrono do TAP.
\end{itemize}

\subsection{TAP Controller}

O \textbf{TAP Controller} é uma máquina de estados síncrona que coordena todo o fluxo de operação. Ele controla:

\begin{itemize}
    \item o carregamento de instruções no IR;
    \item a seleção dos Data Registers;
    \item captura e atualização de valores;
    \item habilitação das operações de boundary scan.
\end{itemize}

\subsection{Instruction Register (IR)}

O IR define qual \textbf{Data Register (DR)} participará da scan-chain. Exemplos de instruções:

\begin{itemize}
    \item \textbf{IDCODE}: retorna o identificador do chip;
    \item \textbf{BYPASS}: reduz a cadeia para 1 bit, útil em cadeias multiprocessadas;
    \item \textbf{EXTEST / SAMPLE}: utilizam o Boundary Scan.
\end{itemize}

\subsection{Boundary Scan}

As células de boundary scan são inseridas entre o pino físico e a lógica interna do chip. Elas permitem:

\begin{itemize}
    \item Capturar estados dos pinos sem interrupção do funcionamento normal;
    \item Injetar valores para teste de trilhas e conexões externas;
    \item Testar placas com encapsulamentos BGA, onde sondagem física é inviável.
\end{itemize}

Esse mecanismo é fundamental em \textbf{DFT — Design for Testability}.

\section{SWD: Diferenças e Funcionamento}

O \textbf{Serial Wire Debug (SWD)} é uma alternativa desenvolvida pela ARM para substituir o JTAG em microcontroladores Cortex.  
Enquanto o JTAG possui foco em testabilidade e depuração, o SWD concentra-se exclusivamente na depuração da CPU.

\subsection{Características do SWD}

\begin{itemize}
    \item Utiliza apenas \textbf{dois fios}: SWCLK e SWDIO.
    \item Mantém as mesmas capacidades de depuração do JTAG.
    \item Permite leitura/escrita de registradores, memória e periféricos.
    \item Permite controle total da CPU: breakpoints, \textit{halt}, \textit{step}, \textit{reset}.
    \item Não possui boundary scan.
\end{itemize}

\section{Aplicação prática do SWD com J-Link v9 no STM32F072RB}

A seguir apresenta-se a demonstração prática utilizando o J-Link v9 e o microcontrolador STM32F072RB.

\subsection{Pinout JTAG do J-Link v9}

\begin{verbatim}
                        ┌───────────┬
          Vref 3.3      │  1  |  2  │   Vref 3.3
         nTRST   ---    │  3  |  4  │   GND
         TDI     ---    │  5  |  6  │   GND
         RMS     SWIO   │  7  |  8  │   GND
         TCK     SWCK   │  9  | 10  │   GND
         RTCK    ---    │ 11  | 12  │   GND
         TDO     SWO    │ 13  | 14  │   GND
            RESET       │ 15  | 16  │   GND
         DBGRO   ---    │ 17  | 18  │   GND
           5v supply    │ 19  | 20  │   GND
                        └───────────┴
\end{verbatim}

\subsection{Pinout SWD da placa STM32 (CN4)}

\begin{verbatim}
Nucleo CN4         J-Link/SWD
Pin 2 (TCK)  <-->  Pin 9  (SWCLK / TCK)
Pin 3 (GND)  <-->  Pin 4  (GND)
Pin 4 (TMS)  <-->  Pin 7  (SWDIO / TMS)
Pin 5 (NRST) <-->  Pin 15 (nRST)
Pin 6 (SWO)  <-->  Pin 13 (SWO / TDO)
Vdd          <-->  Pin 1  (VTref)
\end{verbatim}

\subsection{Configuração do J-Link}

\begin{verbatim}
JLinkExe
Device> STM32F072RB
TIF> S
Speed> 4000
\end{verbatim}

\subsection{Identificação do núcleo ARM}

\begin{verbatim}
mem32 0xE000ED00 1
E000ED00 = 410CC200

Saida:
- 0x41 → ARM  
- 0x0C → Cortex-M0  
- 0xC2 → Revisao  
- 0x00 → Variante  
\end{verbatim}

\subsection{Status do módulo de debug (DHCSR)}

\begin{verbatim}
halt
mem32 0xE000EDF0 -> 00030003

go
mem32 0xE000EDF0 -> 01000001
\end{verbatim}

\subsection{Tamanho da Flash}

\begin{verbatim}
mem32 0x1FFFF7CC 1
1FFFF7CC = FFFF0080

Saida:
- Bits superiores: reservados  
- Bits 15:0 = 0x0080 = **128 KB**
\end{verbatim}

\subsection{Unique ID}

\begin{verbatim}
mem32 0x1FFFF7AC 3
1FFFF7AC = 003B004D 53465710 20343231
\end{verbatim}

\subsection{Leitura de registradores da CPU}

\begin{verbatim}
regs
PC, LR, SP, R0..R12, XPSR...
\end{verbatim}

\subsection{Leitura da RAM}

\begin{verbatim}
mem32 0x20001000 1
20001000 = 12345BD8
\end{verbatim}

\subsection{Pausa da CPU}

\begin{verbatim}
halt
20001000 = 12345BFA
20001000 = 12345BFA
\end{verbatim}

\subsection{Escrita manual na RAM}

\begin{verbatim}
w4 0x20001000 00000000
20001000 = 00000000
\end{verbatim}

\section{Conclusão}

O JTAG permanece essencial para testabilidade estrutural, principalmente pelo uso do boundary scan.  
O SWD, por outro lado, simplifica a depuração de microcontroladores ARM ao reduzir o número de pinos, mantendo as mesmas capacidades de acesso ao núcleo.

O trabalho integra conceitos de DFT, depuração moderna e uso real de ferramentas profissionais como o J-Link v9, consolidando teoria e prática de maneira eficiente.

\end{document}
